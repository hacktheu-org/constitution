\documentclass[12pt]{article}
\usepackage{indentfirst}
\usepackage{tikz}
\usetikzlibrary{arrows}
\usepackage{enumitem}
\usepackage{lipsum,mathptmx}
\usepackage[margin=1in]{geometry}
\usepackage[T1]{fontenc}
\usepackage{baskervillef}
\usepackage[varqu,varl,var0]{inconsolata}
\usepackage[scale=.95,type1]{cabin}
\usepackage[baskerville,vvarbb]{newtxmath}
\usepackage[cal=boondoxo]{mathalfa}
\usepackage{graphicx}
\usepackage{titlesec}
\linespread{1.5}
\pagenumbering{gobble}

\setcounter{secnumdepth}{4}

\newcommand{\linia}{\rule{\linewidth}{0.5pt}}

% my own titles & section headings
\makeatletter
\renewcommand{\maketitle}{
\begin{center}
\vspace{2ex}
{\huge \textsc{\@title}}
\vspace{1ex}
\\
\linia\\
\@author \hfill \@date
\vspace{4ex}
\end{center}
}

\renewcommand{\@seccntformat}[1]{%
  \ifcsname prefix@#1\endcsname
    \csname prefix@#1\endcsname
  \else
    \csname the#1\endcsname\quad
  \fi}
% define \prefix@section
\newcommand\prefix@section{Article \thesection: }
\makeatother
%%%

\titleformat{\subsection}[runin]
{\normalfont\large\bfseries}{\thesubsection}{1em}{}

\title{Constitution of HackTheU}
\author{Adopted May 26, 2016}
\date{Updated \& Ratified May 29, 2020}

\begin{document}
\begin{center}
\includegraphics[width=.3\linewidth]{emblem}
\end{center}

\maketitle

\section{Name}

\subsection{}
The name of this nonprofit organization \& student group shall be HackTheU.

\section{Purpose}

\subsection{} HackTheU's purpose is to foster learning in the Utah technology community through inclusive collaboration, innovation, and self-improvement.

\subsection{} HackTheU will accomplish this by promoting a welcoming technology community 
within higher education and industry throughout the state of Utah 
by hosting free events, workshops, and hackathons.

\subsection{} The primary event held by HackTheU is an annual hackathon.
A hackathon is an event in which participants spend a set period of time exerting their creativity and determination to create an invention that is either software or hardware.
Typically, these events span two days with the main competition running over the course of 24 hours.
Sponsors from local and national companies come from across the nation to support these events
where they can observe the highly skilled teams first hand.
Many sponsors see it as a chance to provide opportunities for the participants and as an avenue of recruitment.

\section{Membership}

\subsection{} Membership is designated as a non-executive position.
\subsection{} Members have no rights or voting privileges within the organization.
\subsection{} In order to be considered a member,
the student must attend at least one HackTheU sponsored event per semester and officially
endorse the organization through official social media platforms such as
Facebook or the University of Utah's Campus Connect.
\subsection{} Membership is open to any currently enrolled student or employee of the University.
\subsection{} Membership offers access to newsletters, gear, and involvement in the organization of the yearly workshops and hackathons.
\subsection{} Members shall indemnify and hold harmless HackTheU and its directors, officers, sponsors, 
affiliates, and members from and against all allegations, claims, actions, suits,
demands, damages, liabilities, obligations, losses, settlements, judgements, costs
and expenses (including without limitation attorneys' fees and costs) which arise out of, 
relate to or result from any act or omission of HackTheU's directors or members to the fullest extent provided by law.

\section{Meeting}

\subsection{} Organizer meetings are held weekly on Wednesdays from 5:00PM to 6:00PM.
\subsection{} Meetings are closed to members and only board members may attend, unless invited by an organizing member.
\subsection{} Board members are expected to be present for the majority of meetings, 
barring any conflicts approved by the Chief Director.
\subsection{} At the beginning of each University of Utah semester, the board will vote to decide on a regular meeting time and place.
\subsection{} Emergency meetings may be called by the Chief Director at any time, as needed.
Board members will make best effort to attend emergency meetings.
The Chief Director will ensure all board members are notified of emergency meetings.
\subsection{} Quorum is defined as having greater than fifty (50) percent of the Organizing Board present at a meeting.

\section{Events}

\subsection{} All events hosted by HackTheU are free and open to members.
\subsection{} All events hosted by HackTheU will have the explicit goal to further its mission and purpose.
\subsection{} Notifications for events will be sent out via email list, Facebook, and signs posted throughout campus. 
Additionally, marketing will be conducted through forms of media as controlled by
the Marketing Committee as defined below.

\section{Organizing Board}

\subsection{} The Organizing Board will be made of Directors and Board Members --- different and separate from HackTheU members --- who will preside over specific committees that are 
responsible for specific duties. Additional positions can be created and ratified through
the amendment process. Organizing positions may also be eliminated and responsibilities
of organizing board members reassigned by the same amendment process.

\subsection{Roles of the Organizing Board}
\subsubsection{Chief Director}
\noindent The Chief Director is responsible for the general well being and head leadership 
of the organization. 
The Chief Director will handle the bulk of administrative work, 
distribute tasks and responsibilities,
and take the primary lead for the HackTheU hackathon.

\subsubsection{Director}
\noindent Directors will hold formal responsibilities inside the organization and 
head leadership of individual committees. The Director of a committee shall be the chief
point of contact for tasks assigned to the committee. The Director must be a University
of Utah student and is expected to attend all weekly all-hands meetings, director's meetings,
and meetings for their individual committee.
Directors are expected to head a single committee.

\subsubsection{Board Member}
\noindent Board members hold no formal responsibilities aside from attending meetings,
but are encouraged to take on tasks as they see fit. Members will join committee(s)
dedicated to task distribution a help complete tasks the committee is responsible for.
Members can be students of any university. Members are expected to attend 
the majority of the semester's all-hands meetings
and meetings for their committee(s). If a Board Member cannot attend a meeting, they
should notify their committee Director. Members are expected to be a part of at least
one committee.

\subsection{} In the case of vacancies, a committee nomination will be made and the notified
individuals will be interviewed within two weeks. The Chief Director will be in charge
of vacant, necessary roles during the search for a candidate. 

\section{Elections}

\subsection{} Each Chief Director for the following year will be nominated by the Chief Director from 
the current active directors who will be continuing to serve the organization.
If this is not possible, nominations for Chief Director 
will be collected from the Organizing Board.
The Chief Director may continue to serve as Chief Director for the following year 
if affirmed by the Organizing Board as detailed in section 7.2.

\subsection{} The nominee for Chief Director must be approved by the Organizing Board
through vote. The nominee must receive greater than fifty (50) percent of votes during
a regular meeting with quorum.

\subsection{} The application process for Directors and Organizing Board Members 
will begin in January and extend for at least two weeks. 

\subsection{} A concise application will be released for interested applicants to complete at the start of Spring Semester.

\subsection{} The current Organizing Board will decide by majority vote who to appoint
as directors and members.

\subsection{Board Member Removal} Organizing Board Members may be removed by the described process:

\subsubsection{} 
\noindent A Board Member may be suggested for removal by their committee Director 
if the member is uncommunicative with all other organizing members and directors for
more than 168 hours continuously.

\subsubsection{} 
\noindent The Chief Director will attempt to schedule an intervention with 
the member who has been suggested for removal. The Chief Director will attempt to
mediate any concerns by the member and attempt to re-engage them and re-assign
their responsibilities and tasks to the member's comfort. 

\subsubsection{} 
\noindent If the Board Member cannot be re-engaged via intervention,
they will be removed if the motion receives greater than fifty (50) percent vote from 
the Organizing Board during a regular meeting with quorum. 

\subsection{} Committee directors may be removed by the same process 
as Board Members if either fifty
(50) percent of their committee members vote to suggest the director for removal, 
or the Chief Director suggests the committee director for removal. 

\subsection{} The Chief Director may be removed and replaced by the same process
if a committee director suggests 
that they should be removed, and an intervention is scheduled with a secondary 
committee director.

\subsection{} Grounds for removal include the inadequate fulfillment of 
aforementioned duties or the
violation of regulations described by University policies or any federal, 
state, or local laws. Violation to University policies or federal, state, or local laws
will result in immediate removal of a member, Board Member, Committee Director, or
Chief Director.

\section{Finance}

\subsection{} All funds will be stored in the University Credit Union account and 
monitored on a monthly basis by the Chief Director.

\subsection{} Income for the organization will primarily come from sponsorships from
supporting corporations.

\subsection{} Monetary funds will only be utilized for the explicit purpose designated in 
the invoice or statement for which the funds are received.

\subsection{} Improper use or management of funds will result in the immediate removal
of the individual(s), with potential consequence of legal action.

\subsection{} In the case of removal, the organizing committee will appoint a new
candidate for the position by the nomination and majority vote procedures described in
section 6.3.

\subsection{} All financial transactions --- income or expenses --- must be conducted
with respect to HackTheU's mission and must seek to preserve its 501(c)(3) status.

\subsection{} The Chief Director will be expected to file all necessary paperwork 
(including taxes, permits, and contract renewals) for the organization.

\section{Committees}

\subsection{} Committees which are currently formed are subject to change due to 
needs in organizing the HackTheU hackathon. For the Spring 2020 to Fall 2021 operational
year, working committees are:

\subsubsection{Logistics}

\noindent The Logistics committee will handle all logistical aspects of event planning,
including but not limited to venue scheduling, food and catering, judging and prizes,
and event registration.

\subsubsection{Technology}

\noindent The Technology committee will handle all technical aspects of HackTheU's 
organizational needs, including but not limited to website management and design,
judging software design, and registration platform management.

\subsubsection{Marketing}

\noindent The Marketing committee will design the general aesthetic for HackTheU and 
fulfill all marketing needs, including managing social media, publishing
promotional videos, publishing print advertising material, and designing event
merchandise and swag.

\section{Affiliations}

\subsection{} HackTheU is currently affiliated with the University of Utah
and no other institutions.

\section{Advisor}

\subsection{} John Melchi, a faculty member in the School of Computing, is the 
advisor for this organization.

\subsection{} Cesar Sanchez, a manager of the Sorenson Center for Innovation, is the
liaison for the school of business to HackTheU.

\section{Ratification}

\subsection{} The constitution will be ratified by membership on the final day of election.

\section{Amendments}

\subsection{} Amendments proposed will be ratified through a bicameral process.
An amendment proposed by the Chief Director must receive greater than fifty percent approval
from the Organizing Board and a veto by the Chief Director can be over-ridden by a 
two-thirds majority vote from the Organizing Board.

\subsection{} Amendments to the Constitution must be submitted to and approved by the 
Department of Student Leadership \& Involvement before they become effective.

\section{Bylaws}

\subsection{} Rules and Regulations will follow the same procedure as the Amendment 
process and can be approved in bulk or through Amendment.

\section{Dissolution}

\subsection{} Upon dissolution of the organization, all assets will be donated to 
a nonprofit organization designated by a majority vote by the current directors
or an organization chosen by the School of Computing of the University of Utah.

\section{Address and Contact}

\noindent The School of Computing of the University of Utah

\noindent 50 S Central Campus Dr Room 3190

\noindent Salt Lake City, UT 84112

\noindent Email: info@hacktheu.org

\noindent Website: www.hacktheu.org

\section{Compliance}

\subsection{} We agree to abide by all regulation described in the 
Guidelines for Recognized Student Organizations, 
all University policies, in addition to all federal, state, and local laws.

\section{Nondiscrimination \& Accessibility Statement}

\noindent This organization shall not discriminate on the basis of race, color, religion, 
national origin, sex, sexual orientation, gender identity/expression, age, or status
as an individual with disability, or as a protected veteran, and any other status protected
by applicable state or federal law. (University of Utah, Policy 6-400, Section II-E). \\

\noindent Additionally, the University endeavors to provide reasonable accommodations
and to ensure equal access to qualified persons with disabilities. Inquired concerning 
perceived discrimination or requests for disability accommodations may be referred to the 
University's Title IX/ADA/Section 504 Coordinator: \\

\noindent Director, Office of Equal Opportunity and Affirmative Action

\noindent 201 South Presidents Circle, Rm. 135

\noindent Salt Lake City, UT 84112

\noindent 801-581-8365 (voice/tdd)

\noindent 801-585-5746 (fax)

\noindent www.oeo.utah.edu
\end{document}
